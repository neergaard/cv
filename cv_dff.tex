%%%%%%%%%%%%%%%%%%%%%%%%%%%%%%%%%%%%%%%%%
% Medium Length Professional CV
% LaTeX Template
% Version 2.0 (8/5/13)
%
% This template has been downloaded from:
% http://www.LaTeXTemplates.com
%
% Original author:
% Trey Hunner (http://www.treyhunner.com/)
%
% Important note:
% This template requires the resume.cls file to be in the same directory as the
% .tex file. The resume.cls file provides the resume style used for structuring the
% document.
%
%%%%%%%%%%%%%%%%%%%%%%%%%%%%%%%%%%%%%%%%%

%----------------------------------------------------------------------------------------
%	PACKAGES AND OTHER DOCUMENT CONFIGURATIONS
%----------------------------------------------------------------------------------------

\documentclass{resume} % Use the custom resume.cls style
\usepackage{microtype}
% \usepackage[left=1.5cm,top=1.5cm,right=1.5cm,bottom=1.5cm]{geometry} % Document margins
% \usepackage[left=0.4 in,top=0.4in,right=0.4 in,bottom=0.4in]{geometry} % Document margins
\usepackage[left=1cm,top=1cm,right=1cm,bottom=1cm]{geometry} % Document margins
\usepackage{hyperref}

\name{Alexander Neergaard Olesen} % Your name
\address{Nørrebrogade 10c, 3. 4 \\ 2200 København N} % Your address
% \address{https://dk.linkedin.com/in/alexanderneergaard} % Your secondary addess (optional)
\address{+45 29840968 \\ alexander.neergaard@gmail.com} % Your phone number and email
% \address{https://dk.linkedin.com/in/alexanderneergaard} % Your secondary addess (optional)
% \address{https://dk.linkedin.com/in/alexanderneergaard} %

\begin{document}

%%----------------------------------------------------------------------------------------
%%	SUMMARY
%%----------------------------------------------------------------------------------------
%
%\begin{rSection}{Summary}
%
%Biomedical engineer from the Technical University of Denmark looking for new opportunities within research and development. Experienced in processing and analysing biomedical signals and images, and designing advanced machine learning algorithms for biomedical applications.
%
%\end{rSection}

%%----------------------------------------------------------------------------------------
%%	COVER LETTER
%%----------------------------------------------------------------------------------------
%
%% To remove the cover letter, comment out this entire block
%
%\recipient{Dr. Emmanuel Mignot}{Stanford Center for Sleep Sciences and Medicine} % Letter recipient
%\date{\today} % Letter date
%\opening{\textsc{Motivational letter}} % Opening greeting
%\closing{Regards,} % Closing phrase
%\enclosure[I have attached the following]{\\Academic resume\\Bachelor degree diploma\\List of grades} % List of enclosed documents
%
%\makelettertitle % Print letter title
%
%Hej
%
%\makeletterclosing % Print letter signature


%----------------------------------------------------------------------------------------
%	EDUCATION SECTION
%----------------------------------------------------------------------------------------
% \clearpage
\begin{rSection}{Education and research experience}
	
	{\bf Technical University of Denmark} \hfill {2016--2020} \\
	PhD, Biomedical Engineering. 
	Thesis title: Deep Learning Methods for Clinical Sleep Analysis \hfill {Kgs. Lyngby, DK}
	
	{\bf Stanford University} \hfill {2017--2019} \\
	Visiting student researcher hosted by Professor Emmanuel Mignot, MD, PhD \hfill {Palo Alto, CA, USA}
	
	{\bf Technical University of Denmark} \hfill {2013--2016} \\
	MScEng, Biomedical Engineering \hfill {Kgs. Lyngby, DK}
	
	{\bf Stanford University} \hfill {2014} \\
	Visiting student researcher hosted by Professor Emmanuel Mignot, MD, PhD \hfill {Palo Alto, CA, USA}	
	
	{\bf Technical University of Denmark} \hfill {2010--2013} \\
	BScEng, Biomedical Engineering \hfill {Kgs. Lyngby, DK}
	
% 	\begin{rSubsection}{Technical University of Denmark}{2013--2016}{MScEng, Biomedical Engineering}{Kgs. Lyngby, DK}
% 	    \item MScEng thesis title: Electrooculography-based Detection and Characterisation of\\Sleep Stages in Patients with Narcolepsy. Grade: 12/12.
% 		\item Selected coursework: advanced machine learning, advanced signal processing with \\biomedical applications, advanced physiological modelling,\\ biomedical product development, pathophysiology. GPA: 10.3/12.
% 	\end{rSubsection}
	
% 	\begin{rSubsection}{Stanford University}{Fall 2014}{Visiting student researcher}{Palo Alto, CA, USA}
% % 		\item 5-month rotation at the Stanford Center for Sleep Sciences and Medicine under \\Professor Emmanuel Mignot, MD, PhD.
% 		\item Selected coursework: experimental design.
% 	\end{rSubsection}
	
% 	\begin{rSubsection}{Technical University of Denmark}{2010--2013}{BScEng, Biomedical Engineering}{Kgs. Lyngby, DK}
% 	    \item BScEng thesis title: Electrochemical Coulter Counter. Grade 12/12.
% 		\item Selected coursework: human biology and diseases, applied signal processing, biomedical \\instrumentation, introduction to medical imaging, modeling of physiological systems, \\biocompatibility of materials. GPA: 10.5/12.
% 	\end{rSubsection}
	
	%\begin{rSubsection}{Stanford University}{Fall 2014}{Visiting Student Researcher at the Stanford Center for Sleep Sciences and Medicine}{}
	%	\item Carried out research concerning the feasibility of using phasic/tonic muscle activations as \\diagnostic measures in a clinical setting at the Stanford Sleep Clinic.
	%	\item Organized entire trip and research project.
	%	\item Raised a total of DKK170.000 as funding.
	%\end{rSubsection}
	
	%{\bf Stanford University} \hfill {Fall 2014} \\ 
	%Visiting Student Researcher at the Stanford Center for Sleep Sciences and Medicine. Researched in nocturnal muscle activity in narcoleptic patients. Specifically investigating the feasibility of using phasic/tonic muscle activations as diagnostic measures in a clinical setting at the Stanford Sleep Clinic.
	
% 	{\bf Technical University of Denmark} \hfill {2010--2013} \\ 
% 	BScEng, Biomedical Engineering. GPA: 10.5/12 \smallskip \\
% 	Selected coursework: human biology and diseases, applied signal processing, biomedical \\instrumentation, introduction to medical imaging, modeling of physiological systems, \\biocompatibility of materials.
	
	%{\bf Technical University of Denmark} \hfill {June 2013} \\ 
	%BSc in Biomedical Engineering. GPA: 10.6
	
% 	{\bf Gentofte HF} \hfill {August 2009} \\ 
% 	Supplementary course, B-level chemistry. Grade: 12/12
	
% 	{\bf Nærum Gymnasium} \hfill {2006--2009} \\
% 	A-levels in Mathematics, Music and French, B-level in Physics. GPA: 11.8/12
	
\end{rSection}


%----------------------------------------------------------------------------------------
%	WORK EXPERIENCE SECTION
%----------------------------------------------------------------------------------------

\begin{rSection}{Employment history}

    {\bf Somnoscient} \hfill {May 2020--present} \\
    Research Engineer, Owner \hfill {Copenhagen, DK}
    
    {\bf Technical University of Denmark} \hfill {2016--2020} \\
    PhD student in the Department of Health Technology \hfill {Kgs. Lyngby, DK}
    
    {\bf Trackman} \hfill {2016} \\
    Development Engineer \hfill {Vedbæk, DK}
    
    {\bf Cathvision} \hfill {2016} \\
    Development Engineer (internship) \hfill {Copenhagen, DK}
    
    {\bf Oticon} \hfill {2015--2016} \\
    Student assistant \hfill {Smørum, DK}
    
    {\bf Novo Nordisk} \hfill {2013--2014} \\
    Student assistant \hfill {Smørum, DK}
    
    {\bf Polyteknisk Forening} \hfill {2012--2013} \\
    Student tutor \hfill {Kgs. Lyngby, DK}
    
    {\bf Technical University of Denmark} \hfill {2012--2015} \\
    Teaching assistant, various courses in the Department of Electrical Engineering \hfill {Kgs. Lyngby, DK}
    

%     \begin{rSubsection}{Somnoscient}{May 2020--present}{Research Engineer (Owner)}{Copenhagen, DK}
% % 		\item Responsible for designing, implementing and testing software and data analysis solutions \\of biomedical and electrophysiological signals for clinical applications.
% 	\end{rSubsection}
	
% 	\begin{rSubsection}{Department of Health Technology, DTU}{Dec 2016--Sep 2020}{PhD student}{Kgs. Lyngby, DK}
% % 		\item Development of clinical support systems for fully automated sleep analysis using biomedical\\signal processing and machine learning technologies.
% % 		\item Co-supervisor of several student projects resulting in several published or in-press \\ research papers, conference papers and abstracts within the field of automated sleep analysis.
% % 		\item Part of the STAGES study, the single largest collection of 30,000 sleep studies with \\ associated clinical and biological data samples, where I have been responsible for the \\ design, development and validation of several data processing and analysis pipelines.
% % 		\item Research collaboration between DTU, Rigshospitalet and Stanford University.
% 	\end{rSubsection}
	
% 	\begin{rSubsection}{TrackMan}{Jul 2016--Dec 2016}{Development Engineer}{Vedbæk, DK}
% % 		\item Design and implementation of algorithms for Doppler radar processing and analysis\\of putting strokes used in the Trackman IV golf performance tracker.
% 	\end{rSubsection}
	
% 	\begin{rSubsection}{CathVision}{May 2016--Jul 2016}{Development Engineer (internship)}{Copenhagen, DK}
% % 		\item Testing and characterization of CathVision ablation equipment.
% % 		\item Development of interfaces between CathVision and third-party equipment.
% % 		\item Development of custom digital signal processing algorithms for real-time ECG monitoring.
% 	\end{rSubsection}

% 	\begin{rSubsection}{Oticon}{Mar 2015--Jan 2016}{Student assistant}{Smørum, DK}
% % 		\item Designing and executing device tests for various hearing aid auxilliary devices in \\the ConnectLine series.
% % 		\item Various ad hoc tasks.
% 	\end{rSubsection}
	
% 	\begin{rSubsection}{Department of Electrical Engineering, DTU}{Sep 2012--May 2015}{Teaching assistant}{Kgs. Lyngby, DK}
% % 		\item Grading assignments and overseeing lab exercises in multiple courses (7 in total) in\\ analog and digital signal processing and analysis, physics, electronics, mathematics, \\MATLAB programming and medical imaging. 
% % 		\item Development of a MATLAB toolbox used in medical imaging course. 
% % 		\item Supervision of MR, CT, PET and ultrasound recordings of phantoms \\containing biological tissues.
% 		%\item Graduate course Applied Signal Processing, Spring 2014. Graded assignments and oversaw lab exercises as well as classroom tutorials.
% 		%\item Undergraduate course Continuous Time Signals and Linear Systems, Spring 2013/2014.
% 		%\item Undergraduate course Introduction to Medical Imaging, Autumn 2013. Primary role was to oversee classroom tutorials, grade student assignments and supervise MR/CT/PET and ultrasonography recordings of phantoms containing biological tissues for the students. Also further developed upon a MATLAB toolbox used in the course for medical image processing and analysis.
% 		%\item Undergraduate courses Applied Engineering Mathematics I, Physics I, Electric Circuits I, and Introductory Programming with MATLAB, Autumn 2012. 
% 	\end{rSubsection}
	
	%------------------------------------------------
	
	%	\begin{rSubsection}{Købmanden i Søllerød}{December 2009 - August 2014}{Assistant Manager}{Søllerød}
	%		\item Managing a medium sized supermarket, including receiving and stocking of \\goods, daily accounting, etc.
	%	\end{rSubsection}
	
	%------------------------------------------------
	
% 	\begin{rSubsection}{Novo Nordisk}{Jun 2013--Jul 2014}{Student assistant}{Måløv, DK}
% % 		\item Assisting on weekends and certain weekdays in a class II laboratory \\performing lab tasks such as cell medium replacement and cell fixation.
% 	\end{rSubsection}
	
	%------------------------------------------------
	
% 	\begin{rSubsection}{Polyteknisk Forening}{Feb 2012--Feb 2013}{Student tutor}{Kgs. Lyngby, DK}
% % 		\item Freshman tutor for new students in the Biomedical Engineering programme. 
% % 		\item Introduced students to academic and social life on campus, and arranged social \\events such weekly get-togethers and bowling nights.
% 	\end{rSubsection}
	
\end{rSection}


%----------------------------------------------------------------------------------------
%	TECHNICAL STRENGTHS SECTION
%----------------------------------------------------------------------------------------

\begin{rSection}{Technical skills}
% 	\begin{rSubsection}{Programming languages}{}{}{}
% 		\item MATLAB (advanced), Python (Advanced), C++ (basic), R (basic).
% 	\end{rSubsection}
	
% 	\begin{rSubsection}{Languages}{}{}{}
% 		\item danish (native), english (fluent), french (basic), german (basic).
% 	\end{rSubsection}
	
	\begin{tabular}{ @{} >{\bfseries}l @{\hspace{6ex}} l }
	Programming languages & Python, MATLAB, C++, R. \\
	Machine learning libraries & PyTorch, Keras, TensorFlow, NumPy, Pandas, scikit-learn. \\
	Developer tools & UNIX shell/bash, git, HPC systems, \LaTeX. \\
	Operating systems & Linux (Ubuntu, CentOS, Manjaro, Arch), Mac OS X, Microsoft Windows \\
	Languages & danish (native), english (fluent), french (basic), german (basic).
	\end{tabular}
	
\end{rSection}


%----------------------------------------------------------------------------------------
%	SCHOLARSHIPS AND GRANTS
%----------------------------------------------------------------------------------------

\begin{rSection}{Funding and awards}
    
    Best poster award: 37th National Meeting on Biomedical Engineering, DMTS19 (DKK 1.000) \hfill {2019} \\
    Travel grant: Otto Mønsteds Fond (DKK 7.500) \hfill {2019} \\
    Travel grant: Otto Mønsteds Fond (DKK 7.500) \hfill {2018} \\ 
    Various travel grants for PhD research stay at Stanford University (total DKK 362.500) \hfill {2017} \\
    % Jorcks Rejselegat: Reinholdt W. Jorck og Hustrus Fond (DKK 100.000) \hfill {2017} \\ 
    % PhD research stipend: Otto Mønsteds Fond (DKK 75.000) \hfill {2017} \\ 
    % IT travel stipend: Stibo Fonden (DKK 68.000) \hfill {2017} \\ 
    % PhD research stipend: Knud Højgaards Fond (DKK 50.000) \hfill {2017} \\ 
    % PhD research stipend: Augustinus Fonden (DKK 39.500) \hfill {2017} \\ 
    % Vera og Carl Johan Michaelsens Legat (DKK 30.000) \hfill {2017} \\ 
    Travel grant: Otto Mønsteds Fond (DKK 9.076) \hfill {2016} \\ 
    Various travel grants for MScEng research stay at Stanford University (total DKK 141.500) \hfill {2014} \\
    % Travel grant: Augustinus Fonden (DKK 30.000) \hfill {2014} \\ 
    % Travel grant: Ellab Fonden (DKK 25.000) \hfill {2014} \\ 
    % Travel scholarship: DTU (DKK 20.000) \hfill {2014} \\ 
    % Travel grant: Tranes Fond (DKK 00.000) \hfill {2014} \\ 
    

    % \begin{rSubsection}{Best poster award, 3rd place}{2019}{}{}
    % \item 37th National Meeting on Biomedical Engineering, DMTS19, DKK 1.000
    % \end{rSubsection}
    
    % \begin{rSubsection}{Travel grant for conference attendance at EMBC'19, Berlin, Germany}{2019}{}{}
    % \item Otto Mønsteds Fond, DKK 7.500
    % \end{rSubsection}
    
    % \begin{rSubsection}{Travel grant for conference attendance at EMBC'18, Honolulu, HI, USA}{2018}{}{}
    % \item Otto Mønsteds Fond, DKK 7.500
    % \end{rSubsection}
    
    % \begin{rSubsection}{PhD research stay at Stanford University}{2017}{DKK 362.500 in total}{}
    % \item Reinholdt W. Jorck og Hustrus Fond, DKK 100.000
    % \item Otto Mønsteds Fond, DKK 75.000
    % \item Stibo Fonden IT travel stipend, DKK 68.000
    % \item Knud Højgaards Fond, DKK 50.000
    % \item Augustinus Fonden, DKK 39.500
    % \item Vera og Carl Johan Michaelsens Legat, DKK 30.000
    % \end{rSubsection}
    
    % \begin{rSubsection}{Travel grant for conference attendance at EMBC'16, Orlando, FL, USA}{2016}{}{}
    % \item Otto Mønsteds Fond, DKK 9.076
    % \end{rSubsection}
    
    % \begin{rSubsection}{Graduate studies research stay at Stanford University}{2014}{DKK 141.500 in total}{}
    % \item Augustinus Fonden, DKK 30.000
    % \item Ellab Fonden, DKK 25.000
    % \item DTU Travel Scholarship, DKK 20.000
    % \item Tranes Fond, DKK 20.000
    % \item Dansk Tennis Fond, DKK 12.000
    % \item Direktør Einar Hansen og hustru fru Vera Hansens Fond, DKK 10.000
    % \item Otto Mønsteds Fond, DKK 10.000
    % \item Oticon Fonden, DKK 9.500
    % \item IDAs og Berg-Nielsens Studie- og Støttefond, DKK 5.000
    % \end{rSubsection}
    
    
% 	\begin{rSubsection}{Otto Mønsteds Fond}{2018}{DKK 7.500 for conference attendance at EMBC'18, Honolulu, HI, USA}{}
% 	\end{rSubsection}
% 	\begin{rSubsection}{Reinholdt W. Jorck og Hustrus Fond}{2017}{DKK 100.000 for PhD studies at Stanford University.}{}
% 	\end{rSubsection}
% 	\begin{rSubsection}{Knud Højgaards Fond}{2017}{DKK 50.000 for PhD studies at Stanford University.}{}
% 	\end{rSubsection}
% 	\begin{rSubsection}{Vera og Carl Johan Michaelsens Legat}{2017}{DKK 30.000 for PhD studies at Stanford University.}{}
% 	\end{rSubsection}
% 	\begin{rSubsection}{Stibo-Fonden IT travel stipend}{2017}{DKK 68.000 for PhD studies at Stanford University.}{}
% 	\end{rSubsection}
% 	\begin{rSubsection}{Otto Mønsteds Fond}{2017}{DKK 75.000 for PhD studies at Stanford University.}{}
% 	\end{rSubsection}
% 	\begin{rSubsection}{Otto Mønsteds Fond}{2016}{DKK 9.076 for conference attendance at EMBC'16, Orlando, FL, USA}{}
% 	\end{rSubsection}
% 	\begin{rSubsection}{DTU Travel Scholarship}{2014}{DKK 20.000 for graduate studies at Stanford University.}{}
% 	\end{rSubsection}
% 	\begin{rSubsection}{IDAs og Berg-Nielsens Studie- og Støttefond}{2014}{DKK 5.000 for graduate studies at Stanford University.}{}
% 	\end{rSubsection}
% 	\begin{rSubsection}{Tranes Fond}{2014}{DKK 20.000 for graduate studies at Stanford University.}{}
% 	\end{rSubsection}
% 	\begin{rSubsection}{Dansk Tennis Fond}{2014}{DKK 12.000 for graduate studies at Stanford University.}{}
% 	\end{rSubsection}
% 	\begin{rSubsection}{Augustinus Fonden}{2014}{DKK 30.000 for graduate studies at Stanford University.}{}
% 	\end{rSubsection}
% 	\begin{rSubsection}{Oticon Fonden}{2014}{DKK 9.500 for graduate studies at Stanford University.}{}
% 	\end{rSubsection}
% 	\begin{rSubsection}{Ellab Fonden}{2014}{DKK 25.000 for graduate studies at Stanford University.}{}
% 	\end{rSubsection}
% 	\begin{rSubsection}{Otto Mønsteds Fond}{2014}{DKK 10.000 for graduate studies at Stanford University.}{}
% 	\end{rSubsection}
	
\end{rSection}

%----------------------------------------------------------------------------------------


%----------------------------------------------------------------------------------------
%	Review experience
%----------------------------------------------------------------------------------------

\begin{rSection}{Scientific service}

\begin{tabular}{ @{} >{\bfseries}l @{\hspace{6ex}} l }
	Volunteer work & EMBC'19 \\
	Review experience & Fondation Leenaards, IEEE Journal of Biomedical Health Informatics (J-BHI), \\
	& IEEE Access, Scientific Reports, IEEE Transactions on Neural Networks and \\
	& Learning Systems (TNNLS), IEEE Transactions on Biomedial Engineering (TBME).
\end{tabular}

\end{rSection}

%----------------------------------------------------------------------------------------



%----------------------------------------------------------------------------------------
%	RESEARCH PROJECTS SECTION
%----------------------------------------------------------------------------------------

% \begin{rSection}{Research projects}
% 	\begin{rSubsection}{DTU Electrical Engineering, Technical University of Denmark}{2015--2016}{MScEng thesis. Grade: 12/12}{}
% 		\item Project title: Electrooculography-based Detection and Characterisation of Sleep Stages in \\Patients with Narcolepsy.
% 		%		\item Designed a two-component system for automatic detection of sleep stages using eye movement signals and subsequent identification of patients with narcolepsy using a mixture of signal processing and machine learning methods.
% 		\item Acquired skills: research dissemination, project planning and execution, implementation of \\machine learning classification methods, statistical analysis, EOG/EEG signal processing/analysis.
% 		\item Outcomes: published paper in peer-reviewed conference proceedings with poster presentation at \\ the 38th Annual Conference ofthe IEEE EMBS, Orlando, Florida, 2016.
% 	\end{rSubsection}
	
% 	\begin{rSubsection}{DTU Compute, Technical University of Denmark}{2015}{Advanced machine learning course project - received grade: 12/12}{}
% 		\item Project title: Nonparametric Bayesian Analysis of the Mouse Brain Connectome.
% 		\item Acquired skills: Bayesian statistical modeling, research dissemination.
% 		\item Poster presentation at Dansk Medicoteknisk Landsmøde, September 2015.
% 	\end{rSubsection}
	
% 	\begin{rSubsection}{Stanford Center for Sleep Sciences and Medicine, Stanford University}{Fall 2014}{}{}
% 		\item Project title: A comparative study of methods for automatic detection of REM sleep without atonia in patients with narcolepsy using an overnight polysomnogram.
% 		\item Acquired skills: EMG signal processing/analysis, planning, fundraising (DKK170.000 in total).
% 		\item Journal paper published in Sleep Medicine.
% 	\end{rSubsection}
	
% 	\begin{rSubsection}{DTU Nanotech, Technical University of Denmark}{2015}{Bachelor project - received grade: 12/12}{}
% 		\item Project title: Electrochemical Coulter Counter.
% 		\item Acquired skills: Cleanroom fabrication of lab-on-a-chip systems, signal acquisition and processing.
% 		%		\item Designed a polymer-based chip with embedded gold electrodes in order to detect impedance changes caused by the flow of small plastic beads across microscale channels. Afterwards used for a lab-on-a-chip based system to detect Alzheimer proteins in spinal fluid.
% 	\end{rSubsection}
% \end{rSection}


%----------------------------------------------------------------------------------------
%	LIST OF PUBLICATIONS
%----------------------------------------------------------------------------------------
\begin{rSection}{List of Publications, Alexander Neergaard Olesen}
	\medskip
    \mbox{*} shared first authorship
    
    \begin{rSubsection}{2020}{}{}{}
        \item \textbf{A. N. Olesen}, P. J. Jennum, E. Mignot, H. B. D. Sorensen. Automatic sleep stage classification with deep residual networks in a mixed-cohort setting. Sleep, zsaa161, 2020. \textsc{doi}: 10.1093/sleep/zsaa161 \textit{(in press)} \medskip
        
        \item \textbf{A. N. Olesen}, P. Jennum, E. Mignot, H. B. D. Sorensen. Deep transfer learning for improving single-EEG arousal detection. 42nd Annual International Conference of the IEEE Engineering in Medicine and Biology Society (EMBC), Montreal, QC, Canada, 2020, pp. 99-103, \textsc{doi}: 10.1109/EMBC44109.2020.9176723 \medskip
        
        \item A. Ambati, Y.-E. Ju, L. Lin, \textbf{A. N. Olesen}, H. Koch, J. J. Hedou, E. B. Leary, V. P. Sempere, E. Mignot, S. Taheri. Proteomic biomarkers of sleep apnea. \textit{Sleep}, zsaa086, 2020. \textsc{doi}: 10.1093/sleep/zsaa086 \textit{(in press)} \medskip
        
        \item A. Brink-Kjær, \textbf{A. N. Olesen}, P. E. Peppard, K. L. Stone, P. Jennum, E. Mignot, H. B. D. Sorensen. Automatic Detection of Cortical Arousals in Sleep and their Contribution to Daytime Sleepiness. Clinical Neurophysiology, 2020;131:1187-1203. \textsc{doi}: 10.1016/j.clinph.2020.02.027 \medskip
	    
	    \item L. Carvelli, \textbf{A. N. Olesen}, A. Brink-Kjaer, E. B. Leary, P. E. Peppard, E. Mignot, H. B. D. Sorensen, P. Jennum. Design of a deep learning model for automatic scoring of periodic and non-periodic leg movements during sleep validated against multiple human experts. Sleep Medicine, 2020;69:109-119. \textsc{doi}: 10.1016/j.sleep.2019.12.032
	\end{rSubsection}
    
	\begin{rSubsection}{2019}{}{}{}
	    \item \textbf{A. N. Olesen}, S. Chambon, V. Thorey, P. Jennum, E. Mignot, H. B. D. Sorensen. Towards a flexible deep learning method for automatic detection of clinically relevant multi-modal events in the polysomnogram. 2019 IEEE 41th Annual International Conference of the IEEE Engineering in Medicine and Biology Society (EMBC), pp. 556-561, Berlin, Germany, 2019. \textsc{doi}: 10.1109/EMBC.2019.8856570
	\end{rSubsection}
	
	\begin{rSubsection}{2018}{}{}{}
	    \item J. B. Stephansen\mbox{*}, \textbf{A. N. Olesen}\mbox{*}, M. Olsen, et al. Neural network analysis of sleep stages enables efficient diagnosis of narcolepsy. Nature Communications, 9:5229, 2018. \textsc{doi}: 10.1038/s41467-018-07229-3 \medskip
	    
	    \item \textbf{A. N. Olesen}, P. Jennum, P. E. Peppard, H. B. D. Sorensen, E. Mignot. Deep Residual Networks for Automatic Sleep Stage Classification of Raw Polysomnographic Waveforms. 2018 IEEE 40th Annual International Conference of the IEEE Engineering in Medicine and Biology Society (EMBC), pp. 1-4, Honolulu, HI, USA, 2018. \textsc{doi}: 10.1109/EMBC.2018.8513080\medskip
	    
	    \item A. B. Klok\mbox{*}, J. Edin\mbox{*}, M. Cesari, \textbf{A. N. Olesen}, P. Jennum, H. B. D. Sorensen. A New Fully Automated Random-Forest Algorithm for Sleep Staging. 2018 IEEE 40th Annual International Conference of the IEEE Engineering in Medicine and Biology Society (EMBC), pp. 4920–4923, Honolulu, HI, 2018. \textsc{doi}: 10.1109/EMBC.2018.8513413 \medskip
	    
	    \item M. Cesari, J. A. E. Christensen, L. Kempfner, \textbf{A. N. Olesen}, G. Mayer, K. Kesper, W. H. Oertel, F. Sixel-Döring, C. Trenkwalder, H. B. D. Sorensen, and P. Jennum. Comparison of computerized methods for REM sleep without atonia detection. Sleep, Volume 41, Issue 10, zsy133, 2018. \textsc{doi}: 10.1093/sleep/zsy133 \medskip
	    
	    \item \textbf{A. N. Olesen}\mbox{*}, M. Cesari\mbox{*}, J. A. E. Christensen, H. B. D. Sorensen, E. Mignot, and P. Jennum. A comparative study of methods for automatic detection of rapid eye movement abnormal muscular activity in narcolepsy. Sleep Medicine, vol. 44, pp. 97–105, 2018. \textsc{doi}: 10.1016/j.sleep.2017.11.1141
	\end{rSubsection}
	
	\begin{rSubsection}{2016}{}{}{}
	    \item \textbf{A. N. Olesen}, J. A. E. Christensen, H. B. D. Sorensen, and P. J. Jennum. A Noise-Assisted Data Analysis Method for Automatic EOG-Based Sleep Stage Classification Using Ensemble Learning. 2016 IEEE 38th Annual International Conference of the IEEE Engineering in Medicine and Biology Society (EMBC), pp. 3769–3772, Orlando, FL, USA, 2016. \textsc{doi}:  10.1109/EMBC.2016.7591548
	\end{rSubsection}
	
	
% 	\item Olesen, A.N., Cesari, M., Christensen, J.A.E., Sorensen, H.B.D., Mignot, E., and Jennum, P. A comparative study of methods for automatic detection of REM abnormal muscular activity in narcolepsy. \textit{in press}
% 	\item Olesen, A.N., Christensen, J.A.E., Jennum, P. and Sorensen, H.B.D. A Noise-Assisted Data Analysis Method for Automatic EOG-Based Sleep Stage Classification using Ensemble Learning. In \textit{Proceedings of the 38th Annual International Conference of the IEEE Engineering in Medicine and Biology Society}, Orlando, Florida, 2016.
\end{rSection}







%----------------------------------------------------------------------------------------
%	EXAMPLE SECTION
%----------------------------------------------------------------------------------------

%\begin{rSection}{Section Name}

%Section content\ldots

%\end{rSection}

%----------------------------------------------------------------------------------------

\end{document}
